\documentclass[12pt]{article}
\usepackage[utf8]{inputenc}
\usepackage{fancyhdr}
\usepackage{graphicx}
\usepackage{amsmath}
\usepackage{amssymb}
\usepackage{amsfonts}
\usepackage[margin=2.5cm]{geometry}
\usepackage{listings}
\usepackage[svgnames]{xcolor}

\lstset{language=R,
	showspaces=false,
	showstringspaces=false,
	showtabs=false,
	keywordstyle=\color{NavyBlue},
	commentstyle=\color{YellowGreen},
	stringstyle=\color{Crimson}
}

\setlength{\jot}{1.5ex}

\title{Lista 3\\
Econometria\\
Diogo Wolff Surdi}

\begin{document}

\maketitle

\section*{Questão 6}

\subsection*{(a)}
\begin{lstlisting}[language=R]
library(foreign)
library(tidyr)
library(dplyr)
library(lmtest)
library(sandwich)

dados <- read.dta('(...)/fertil2.dta')

NAs <- dados %>%
filter_at(vars(children, age, educ, electric, urban), any_vars(is.na(.)))

dados <- anti_join(dados, NAs)
dados$agesquared <- dados$age^2

reg <- lm(children ~ age+agesquared+educ+electric+urban, dados)
erropadrao <- coef(summary(reg))[,2]

errorobusto <- coeftest(reg, vcov = vcovHC(reg, type="HC0"))[,2]

erros <- cbind(erropadrao, errorobusto)

> erros
erropadrao  errorobusto
(Intercept) 0.2401887611 0.2436830685
age         0.0165082295 0.0191614558
agesquared  0.0002717715 0.0003502695
educ        0.0062966083 0.0063033633
electric    0.0690044704 0.0639041125
urban       0.0465062118 0.0454396178
\end{lstlisting}
O erro usual não necessarimanete é maior (menor) do que o robusto.

\subsection*{(b)}
\begin{lstlisting}[language=R]
library(car)

reg2 <- lm(children ~ age+agesquared+educ+electric+urban+
	spirit+protest+catholic, dados)

nullhp<-c('spirit', 'protest', 'catholic')

> linearHypothesis(reg2, nullhp)
Linear hypothesis test

Hypothesis:
spirit = 0
protest = 0
catholic = 0

Model 1: restricted model
Model 2: children ~ age + agesquared + educ + electric + urban + spirit + 
protest + catholic

Res.Df    RSS Df Sum of Sq      F  Pr(>F)  
1   4352 9176.4                              
2   4349 9162.5  3     13.88 2.1961 0.08641 .
\end{lstlisting}
O p-valor do teste usual é 0.086. Para o teste robusto, temos, utilizando o método do livro:

\begin{lstlisting}[language=R]
u <- resid(reg)

spi <- resid(lm(spirit ~age+agesquared+educ+electric+urban, dados))
pro <- resid(lm(protest ~age+agesquared+educ+electric+urban, dados))
cat <- resid(lm(catholic ~age+agesquared+educ+electric+urban, dados))

p1 <- u*spi
p2 <- u*pro
p3 <- u*cat

magia <- rep(1, 4358)

reg3 <- lm(magia ~ 0+p1+p2+p3)

ssr <- sum(resid(reg3)^2)

> dim(dados)[1]-ssr
[1] 6.465107

> pchisq(dim(dados)[1]-ssr, df=3, lower.tail=FALSE)
[1] 0.0910488
\end{lstlisting}
Logo o p-valor é 0.091. Assim, para ambos os testes, rejeitamos a hipótese ao nível 0.1, mas não a rejeitamos a níveis usuais menores.

\subsection*{(c)}
\begin{lstlisting}[language=R]
fit <- fitted(reg)
fit2 <- fitted(reg2)
fit2sq <- fit2^2

itemc <- lm(resid^2 ~ fit2 + fit2sq)
> itemc

Call:
lm(formula = resid^2 ~ fit2 + fit2sq)

Coefficients:
(Intercept)         fit2       fit2sq  
0.3126      -0.1489       0.2668  

nula <- c('fit2', 'fit2sq')

> linearHypothesis(itemc, nula)
Linear hypothesis test

Hypothesis:
fit2 = 0
fit2sq = 0

Model 1: restricted model
Model 2: resid^2 ~ fit2 + fit2sq

Res.Df   RSS Df Sum of Sq      F    Pr(>F)    
1   4357 76589                                  
2   4355 57436  2     19153 726.11 < 2.2e-16 ***
\end{lstlisting}
Como pode-se ver, rejeitamos a hipótese de que o quadrado do erro não é correlacionado com essas duas variáveis, logo a variância do erro apresenta correlação com elas, e há heteroscedasticidade. 

\subsection*{(d)}
\begin{lstlisting}[language=R]
cont <- dados %>% group_by(age) %>% summarise(contagem = sum(age))                                      
cont$contagem <- cont$contagem/cont$age
plot(cont)
\end{lstlisting}

\begin{center}
	\includegraphics*[scale=0.6]{1.png}
\end{center}
Como pode-se ver, há muito menos dados ao longo do tempo, então afirmo que a variância do erro ser correlacionada com a idade não é algo muito relevante, pois há muito menos dados conforme a idade aumenta, logo é natural que o erro da regressão seja maior.

\end{document}



