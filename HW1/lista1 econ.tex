\documentclass[12pt]{article}
\usepackage[utf8]{inputenc}
\usepackage{graphicx}
\usepackage{amsmath}
\usepackage{amssymb}
\usepackage{amsfonts}
\usepackage[margin=2.5cm]{geometry}
\usepackage{listings}
\usepackage[svgnames]{xcolor}

\lstset{language=R,
	showspaces=false,
	showstringspaces=false,
	showtabs=false,
	keywordstyle=\color{NavyBlue},
	commentstyle=\color{YellowGreen},
	stringstyle=\color{Crimson}
}

\setlength\parindent{0pt} %% Do not touch this

\title{Lista 1} %% Assignment Title

\author{Diogo Wolff Surdi\\ %% Student name
Econometria I\\ %% Code and course name
\textsc{FGV}
}

\date{16 de Março de 2020}

%% %%%%%%%%%%%%%%%%%%%%%%%%%
\begin{document}   
%% %%%%%%%%%%%%%%%%%%%%%%%%%
\maketitle

\section*{Questão 1}

\begin{equation*}
    \begin{gathered}
        Cov(X,Y)=\mathbb{E}[(X-\mathbb{E}[X])(Y-\mathbb{E}[Y])]\\
        =\mathbb{E}[XY-X\mathbb{E}[Y]-\mathbb{E}[X]Y+\mathbb{E}[X]\mathbb{E}[Y]]\\
        =\mathbb{E}[XY]-\mathbb{E}[X]\mathbb{E}[Y]-\mathbb{E}[X]\mathbb{E}[Y]+\mathbb{E}[X]\mathbb{E}[Y]\\
        =\mathbb{E}[XY]-\mathbb{E}[X]\mathbb{E}[Y]
    \end{gathered}
\end{equation*}\\
Se $\mathbb{E}[X]=0$ ou $\mathbb{E}[Y]=0$, então o segundo termo se anula, logo $Cov(X,Y)=\mathbb{E}[XY]$.

\section*{Questão 2}

\begin{align*}
    \mathbb{E}[X]&=\int_x xf_xdx\\
    &=\int_x x\left(\int_y f_{X,Y=y}dy \right) dx\\
    &=\int_x x\left(\int_y f_{X|Y=y}f_{Y=y}dy\right)dx\\
    &=\int_x \int_y xf_{X|Y=y}f_{Y=y}dydx\\
    &=\int_y \int_x xf_{X|Y=y}f_{Y=y}dxdy\\
    &=\int_y\left(\underbrace{\int_x xf_{X|Y=y}dx}_{=\mathbb{E}[X|Y]}\right)f_{Y=y}dy\\
    &=\mathbb{E}[\mathbb{E}[X|Y]]
\end{align*}

\section*{Questão 3}
As propriedades do MQO são linearidade nos parâmetros, aleatoriedade da amostra, esperança condicional do erro nula ($\mathbb{E}[u|x_1,x_2,...,x_n]=0$), e não-existência de colinearidade perfeita.

\subsection*{(a)}
Sabemos que $\hat{\beta}_0=\Bar{y}-\hat{\beta_1}\Bar{x}$ e $\hat{\beta}_1=\frac{\sum_{i=1}^n(x_i-\Bar{x})y_i}{\sum_{i=1}^n(x_i-\Bar{x})^2}$. Começando pelo segundo item, podemos realizar as tranformações:

\begin{equation*}
    \hat{\beta}_1=\frac{\sum_{i=1}^n(x_i-\Bar{x})y_i}{SQT_x}=\frac{\sum_{i=1}^n(x_i-\Bar{x})(\beta_0+\beta_1x_i+u_i)}{SQT_x}
\end{equation*}

O numerador pode ser transformado em:

\begin{equation*}
    \begin{gathered}
        \sum_{i=1}^n(x_i-\Bar{x})\beta_0+\sum_{i=1}^n(x_i-\Bar{x})\beta_1x_i+\sum_{i=1}^n(x_i-\Bar{x})u_i\\
        =\beta_0\sum_{i=1}^n(x_i-\Bar{x})+\beta_1\sum_{i=1}^n(x_i-\Bar{x})x_i+\sum_{i=1}^n(x_i-\Bar{x})u_i
    \end{gathered}
\end{equation*}

Note que $\sum_{i=1}^n(x_i-\Bar{x})=0$ e $\sum_{i=1}^n(x_i-\Bar{x})x_i=\sum_{i=1}^n(x_i-\Bar{x})^2=SQT_x$, logo podemos reescrever a equação como:

\begin{equation*}
    \hat{\beta}_1=\beta_1SQT_x+\sum_{i=1}^n(x_i-\Bar{x})u_i=\beta_1+\frac{1}{SQT_x}\sum_{i=1}^n(x_i-\Bar{x})u_i
\end{equation*}

Com isso, basta tomar a esperança condicional em relação à amostra para encontrar o resultado:

\begin{align*}
    \mathbb{E}[\hat{\beta}_1|x_1,...,x_n]&=\beta_1+\mathbb{E}[\frac{1}{SQT_x}\sum_{i=1}^n(x_i-\Bar{x})u_i|x_1,...,x_n]\\
    &=\beta_1+\frac{1}{SQT_x}\mathbb{E}[\sum_{i=1}^n(x_i-\Bar{x})u_i|x_1,...,x_n]\\
    &=\beta_1+\frac{1}{SQT_x}\sum_{i=1}^n(x_i-\Bar{x})\mathbb{E}[u_i|x_1,...,x_n]\\
    &=\beta_1+\frac{1}{SQT_x}\sum_{i=1}^n(x_i-\Bar{x})\times0\\
    &=\beta_1
\end{align*}

Sabendo que vale para $\hat{\beta}_1$, a prova para $\hat{\beta}_0$ é trivial. Temos que $\Bar{y}=\beta_0+\beta_1\Bar{x}+\Bar{u}$, logo $\hat{\beta}_0=\Bar{y}-\hat{\beta}_1\Bar{x}=\beta_0+(\beta_1-\hat{\beta}_1)\Bar{x}+\Bar{u}$. Com isso:

\begin{align*}
    \mathbb{E}[\hat{\beta_0}|x_1,...,x_n]&=\beta_0+\mathbb{E}[(\beta_1-\hat{\beta}_1)\Bar{x}|x_1,...,x_n)]+\mathbb{E}[\Bar{u}|x_1,...,x_n]\\&
    =\beta_0+\Bar{x}\mathbb{E}[(\beta_1-\hat{\beta}_1)|x_1,...,x_n]\\&=
    \beta_0
\end{align*}

\section*{Questão 4}
Seja $\mathbb{E}[u]=\alpha_0$. Então, basta-se tomar $y=(\beta_0+\alpha_0)+\beta_1x+u-\alpha_0$, gerando novos $\beta_0^*=\beta_0+\alpha_0$ e $u^*=u-\alpha_0$. Note que $\mathbb{E}[u^*]=0$, logo encontramos o resultado esperado.

\section*{Questão 5}

\subsection*{(a)}
O peso com 0 cigarros é 199.77, enquanto que o peso para 20 cigarros é 189.49. Essa difirença indica que mães que fumam cigarros tendem a ter filhos com peso menor ao nascer.

\subsection*{(b)}
Apesar da correlação ser não-nula, podem existir outros fatores correlacionados com a queda no peso dos bebês. O caso pode ser de que mães que fumam durante a gravidez desconhecem os riscos que isso gera, e assim também podem incorrer em outros comportamentos maléficos para a saúde do bebê, sendo que a perda de peso pode ser causada por esses efeitos ulteriores.

\subsection*{(c)}
\begin{equation*}
    125=199.77-0.514x\:\:\:\longrightarrow \;\;\; x \approx 145
\end{equation*}

\subsection*{(d)}
Como há muitas mulheres não-fumantes no grupo, há grande quantidade de variação que não é resultante de fumar, mas que afeta a regressão pois está sendo contabilizada. Com isso, há um forte viés de omissão presente na regressão, explicando o valor exorbitante encontrado no item anterior.

\section*{Questão 6}
As hipóteses são versões n-dimensoniais das propriedades do MQO, somadas à hipótese de homoscedasticidade. O teorema é importante pois afirma que o método dos mínimos quadrados gera o estimador não-enviesado de menor variância.

\section*{Questão 7}

\subsection*{(a)}
O sinal de $\beta_0$ deve ser positivo, pois casas tem valor positivo logo a média deve o ser para qualquer valor das outras variáveis. Como a poluição diminui a qualidade de vida das pessoas, espera-se que $\beta_1$ possua valor negativo.

\subsection*{(b)}
É razoável supor que há menos poluição em bairros nobres, pois pessoas ricas evitariam áreas com essa característica, e moradores de tais localidades possuem maior renda, podendo construir casas com mais quartos. Com isso, tais fatores são negativamente correlacionados. Considerando agora a modelagem sem a segunda variável independente, temos que observar o efeito da omissão dela na estimação.\\\\

Sabemos que $\hat{\beta}_{1}=\frac
{\sum_{i=1}^{n}(x_{i1}-\bar{x}_{1})y_{i}}
{\sum_{i=1}^{n}(x_{i1}-\bar{x}_1)^{2}}$.
Sendo o modelo verdadeiro $y_{i}=\beta_{0}+\beta_{1}x_{i1}+\beta_{2}x_{i2}+u_{i}$, ao substituir no denominador de $\hat{\beta}$ encontramos:

\begin{equation*}
	\sum_{i=1}^{n}(x_{i1}-\bar{x}_{1})y_{i}=
	\beta_{1} \sum_{i=1}^{n}(x_{i1}-\bar{x}_{1})^{2}+
	\beta_{2} \sum_{i=1}^{n}(x_{i1}-\bar{x}_{1})x_{i2}+
	\sum_{i=1}^{n}(x_{i1}-\bar{x}_{1})u_{i}
\end{equation*}

Voltando para $\beta_{1}$, isto é, dividindo a equação por $\sum_{i=1}^{n}(x_{i1}-\bar{x}_1)^{2}$, e tirando o valor esperado, encontramos que:

\begin{equation*}
	\mathbb{E}[\hat{\beta_{1}}]=\beta_{1}+
	\beta_{2}\frac            {\sum_{i=1}^{n}(x_{i1}-\bar{x}_{1})x_{i2}}
							  {\sum_{i=1}^{n}(x_{i1}-\bar{x}_1)^{2}}
\end{equation*}

Pois $\mathbb{E}[u_{i}]=0$. Note que, $\sum_{i=1}^{n}(x_{i1}-\bar{x}_{1})x_{i2}=
\sum_{i=1}^{n}(x_{i1}-\bar{x}_{1})(x_{i2}-\bar{x}_{2})=(n-1)Cov(x_1,x_2)$, logo o sinal da correlação entre os dois é o mesmo do termo da equação. Como visto anteriormente, tal valor será então negativo. Como casas com mais quartos tendem a ser mais caras, podemos supor que $\beta_{2}\textgreater 0$. Com isso, o valor esperado será:

\begin{equation*}
	\mathbb{E}[\hat{\beta_{1}}]=\beta_{1}+
	\beta_{2}\frac            {\sum_{i=1}^{n}(x_{i1}-\bar{x}_{1})x_{i2}}
						      {\sum_{i=1}^{n}(x_{i1}-\bar{x}_1)^{2}}
	\;\textless \; \beta_{1}					      
\end{equation*}

Com isso, podemos afirmar que a regressão \textit{subestima} o valor de $\beta_{1}$ (isso é, a regressão sobrevaloriza o impacto da poluição).

\subsection*{(c)}
Sim, a relação é como esperado. É válido afirmar que $-0.781$ está mais próximo do que $-1.0043$ da elasticidade verdadeira, pois há mais variáveis explicativas não-redundantes, logo a estimação deve estar mais próxima do modelo real.

\section*{Questão 8}

\subsection*{(a)}

\begin{lstlisting}[language=R]
library(foreign)
dados <- read.dta("C:/Users/dwsur/OneDrive/Area de Trabalho/
docs/livros do curso/5 periodo/econometria 1/lista 1/
econometria_2020-master/ceosal2.dta")
mean(dados[,'salary'])
mean(dados[,'ceoten'])
\end{lstlisting}

Os valores encontrados foram salário médio de 865.9 e permanência média de 7.955.\\

\subsection*{(b)}

Gero o vetor de dados da coluna que se encaixam na especificação, e encontro o tamanho dele. No caso, é 5.

\begin{lstlisting}[language=R]
ano0 <- which(dados$ceoten==0)
length(ano0)
\end{lstlisting}

Para encontrar a maior permanência é mais simples. No caso, é 37.

\begin{lstlisting}[language=R]
max(dados$ceoten)
\end{lstlisting}

\subsection*{(c)}
A função $\log (salary)$ já está computada em uma coluna dos dados.

\begin{lstlisting}[language=R]
modelo <- lm(lsalary ~ ceoten, dados)
modelo

Call:
lm(formula = lsalary ~ ceoten, data = dados)

Coefficients:
(Intercept)       ceoten  
6.505498     0.009724  
\end{lstlisting}

Para cada ano de permanência no cargo, o salário aumenta cerca de $0.97\%$.

\section*{Questão 9}

\subsection*{(a)}

\begin{lstlisting}[language=R]
regressao <- lm(price ~ 
	sqrft + bdrms,
	dados)
regressao

Call:
lm(formula = price ~ sqrft + bdrms, data = dados)

Coefficients:
(Intercept)        sqrft        bdrms  
-19.3150       0.1284      15.1982
\end{lstlisting}

A função é então $price=-19.3150+0.1284sqrft+15.1982bdrms+u$.

\subsection*{(b)}
O aumento estimado com um quarto a mais no preço é de 15.19 milhares de dólares.

\subsection*{(c)}
O aumento estimado seria de $15.19+140\times 0.1284\approx 33.17$ milhares de dólares.

\subsection*{(d)}

\begin{lstlisting}[language=R]
summary(regressao)

Call:
lm(formula = price ~ sqrft + bdrms, data = dados)

Residuals:
Min       1Q   Median       3Q      Max 
-127.627  -42.876   -7.051   32.589  229.003 

Coefficients:
Estimate Std. Error t value Pr(>|t|)    
(Intercept) -19.31500   31.04662  -0.622    0.536    
sqrft         0.12844    0.01382   9.291 1.39e-14 ***
bdrms        15.19819    9.48352   1.603    0.113    
---

Residual standard error: 63.04 on 85 degrees of freedom
Multiple R-squared:  0.6319,	Adjusted R-squared:  0.6233 
F-statistic: 72.96 on 2 and 85 DF,  p-value: < 2.2e-16

\end{lstlisting}

O $R^2$ é $0.6319$, logo as variáveis explicam cerca de $63.2\%$ da variação.\\\\

\subsection*{(e)}
\begin{equation*}
	-19.3150+0.1284\times2438+15.1982\times4=354.517
\end{equation*}

\subsection*{(f)}
O resíduo é $\hat{u}=y-\hat{y}=300-354.517=-54.517$, logo, o valor sugere que ele pagou pouco pela casa.

\end{document}